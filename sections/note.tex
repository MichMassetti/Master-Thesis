\chapter{Outcome for individual tools}
\label{ch:Running}

This chapter is moved in the next one, I keep now bc it contains the notes

Fomal specifications, outcomes, time of running, no comparison, single description, outliers (performe very well or very baddly), 
difficulties in the installation and running 


Detalais regarding what I did for running the test, so for each tool I describe the setting for each scripts.
\subsection{SmarTest}
In some cases it takes to long so I added a timer of 320 with the setting 
no reentrancy by definition 
set up in exploit mode + assertion mode 
About Aku it gives half of the result, it states that all the user has to be refunded, but the

\subsection{Slither}
AKU: it detects the presence of arbitrary call but it doesn't detect that can get stucked 
Correct detection of all the reentrancy 
Cover: warning about the comparison based on blocktimestamp 

it has no capability for guess the other vulenrabilities 

\subsection{Mythril}

AkU: nada 
Correct detection of all the reentrancy 
Cover: warning about the comparison based on blocktimestamp 

\subsection{SolcVerify}
Notes:
\begin{itemize}
    \item inheritance allowed 
    \item it worked well with the codes (Solidity Version 7)
    \item the analysed files are still in solidity 
    \item grammar language very intuitive 
    \item it allwos even not flat contract 
\end{itemize}
It works better than Celestial 

\subsection{Celestial}
Notes:
\begin{itemize}
    \item no inheritance
    \item no struct and no array as parameter in function 
    \item total different language 
    \item more grammar for the language 
    \item A lot of grammar 
    \item no address(this), it gives error in fstar
    \item problem with internal functions , fstar could not find the identifier (BZX) 
    \item not so clear which property is violeted  
    \item no loop, neither invariant neither written
\end{itemize}

Uranium: false positive

Spartan: it works good 

bZX: about properties it worked properly, problem with internal functions 

BurgerSwap \& DirtyDogs \& SurgeProcol: no reentrancty bc Calling external contracts are avoided.
Contract Local Reasoning: Calling external contracts
can lead to reentrant behavior where the external contract
calls back into the caller, which is often difficult to reason
about. CELESTIAL disallows such behaviors by checking for
external callback freedom (ECF) [28], [42] which states that
every contract execution that contains a reentrant callback is
equivalent to some behavior with no reentrancy. When this
property holds, it is sufficient to reason about non-reentrant

Aku: it didn't work with the loop so rewrite the contract, but it worked 

Cover Protocol: celestial doesn't recognize memory and storage keyword

\subsection {Manticore}

BurgerSwap \& DirtyDogs \& SurgeProcol: no reentrancy, bc it gives always a fake positive
You can test the Reentrancy just with the tool without specification, runninng Manticore as Scanner 

Spartan: worked well
Uranium: fake positive

Aku: Worked well , it sayes that the rule about claimProject is broken just if everybid is 1 

Cover: doesn't work 
\subsection {Echidna}
Spartan: it worked correctly and it gave even the transaction oreder, tested with assertions and with test Function

Aku: worked well , it found the correct sequence of transaction for block the contract 

Uranium: it worked prorly, it gave the list of actions to do 

Cover Protocol: worked good


BZX: verified with the assertion but it woked proeprly



\subsection{Manticore}

\subsection{Certora}
The time of execution is given by the application
\subsection{Tool without Specifiction}
Mytrhil and Slither 
Slither super fast 
Mythril in case i twas too slow we put a timer of 240 seconds
A timer was add in python modifying the main caller file 

\chapter {cOse}
\subsection{Tools without specifications}
\label{ToolsWithouSpecifications}
\begin{table*}
  \caption{Tools without Specifications}
  \label{tab:Toolsee}
  \begin{tabular}{cc}
  \toprule
    Tools Without Specification\\
    \midrule
    Slither \\
    Mythril \\
    SmarTest \\
    Oyente  \\
    \toprule
    Tools With Specification\\
    \midrule
    Echidna \\
    Manticore  \\
    SolcVerify\\
    \bottomrule
 \end{tabular}
\end{table*}

\section{XSURGE on BSC Chain}   
\label{sec:Exploits:XSURGE}
The \citet{XSurgeWeb}'s whitepaper provides a presentation of the ecosystem.
It is described as a great DeFi investing idea based on proprietary pricing algorithms embedded in the Surge Token Variants' contracts.
Surge Token Variants each have their own Market Maker, allowing them to trade continuously and outlast both 
centralised and decentralised exchanges. 
The strategy is to reward long-term holding by increasing a
holder's claim of the backing asset. Each Surge Token utilizes a built-in contract exchange system that renounces the need for
a traditional liquidity pool. Both assets are stored within the contract itself, 
rather than a liquidity pool pair of the backing asset to the
token using a traditional market maker method for exchange and price calculation.

\subsection{The explpoit}
One of the Surge Token is SurgeBNB, the one which is my focus of analysis.
\citet{XSurgeBNB} explains in deep how the attack to this contract occured. 
The team claimed that the attacker had stolen \$5 million in SurgeBNB through a backdoor vulnerability.
XSURGE stated that a potential security vulnerability in the SurgeBNB contract was discovered on August 16th.

The attack is mabe by 4 main steps:
\begin{enumerate}
    \item the attacker borrow  10,000BNB through flash loans.
    \item Use all the BNB to buy SURGE. According to the current price, 
    the attacker can buy 1,896,594,328,449,690 SURGE
    \item He calls the "sell" function, for selling the obtained SURGE.
    \item The sale function alters the data after the transfer, and the transfer code has a reentrance vulnerability.
    When the attack contract acquires BNB, the period before the SURGE contract's state changes 
    (\refname{lst:SellSURGE}), the attack contract can use the reentrance 
    vulnerability to purchase SURGE again.
\end{enumerate}

\begin{lstlisting} [language={Solidity},caption={Sell function of Surge (SURGE) token.}, label={lst:SellSURGE}]
    function sell(uint256 tokenAmount) public nonReentrant returns (bool) {
        ...
        //The reentrancy 
        (bool successful,) = payable(seller).call{value: amountBNB, gas: 40000}(""); 
        if (successful) {
            // subtract full amount from sender
            _balances[seller] = _balances[seller].sub(tokenAmount, 'sender does not have this amount to sell');
            // if successful, remove tokens from supply
            _totalSupply = _totalSupply.sub(tokenAmount);
        } else {
            revert();
        }
        
        return true;
    }
\end{lstlisting}


The bnb Amount of the contract stays intact, and the total amount of SURGE tokens \texttt{totalSupply}  
has not been updated, because the attack contract spends all of the BNB balance to acquire SURGE
 each time (still remains the quantity before the sell).
As a result, the price of token falls, allowing the attacker to purchase additional SURGE. 

Repeating three times of Round 2 and Round 3 , the attacker accumulates a large amount of SURGE through reentry, and then sells all the SURGE to make a profit.

At the end of this transaction, the attack contract sold 1,864,120,345,279,610,000 SURGE, 
obtained 10327 BNB, and finally the profitable 297 BNB was sent to the attacker's address.

The following are the modifications suggested by the Beosin technical team for this attack:
\begin{itemize}
    \item any transfer operation should be place after the state changes to avoid reentry assaults.
    \item Instead of using "call. value," use transfer or send to transfer. 
\end{itemize}

\subsection{Properties}

This exploit represents a typical case of reentrancy. 

The attacker's strategy involves the function sell, which contains the bug, and then the function purchase. 
After calling the first one and triggering the reentrancy, the malicious fallback implemented by the attacker uses the amount of money for buying more XSURGE tokens. 
At the end of the selling process, the total supply should decrease the amount sold by the user.
But since the attacker called the purchase, the variable is not updated as it was supposed to be. 
Buying the same amount of sold tokens, the value would not change.

We define the property as a postcondition,refered to the function sell(...), which states the variable \_totalSupply is decreased of the amount sold by the user, then tokenAmount.

\begin{equation}
    \begin{split}
        \_totalSupply <=old\_value\_of(\_totalSupply)
    \end{split}
\end{equation}

The properties can be even expressed like a an invariant, stating that the sum of the single balances cannot exceedes the variable \_totalSupply.
\begin{equation}
    \begin{split}
        sum_of_uint(_balances) <= _totalSupply
    \end{split}
\end{equation}


\section{Weaknesses and Strenghts}
%% What I wrote in the table 
%% I talk even about the installation and the running phase

We present the 
\begin{table*}
    \caption{Weaknesses \& Drawbacks}
    \label{tab:Weaknesses}
    \begin{tabular}{cc}
    \toprule
        Tools  &  Weaknesses \& Drawbacks \\
        \midrule
        Manticore & Reentrancy is not dectected by properties property 
        based execution, very slow \\
        SmartTest & Reentrancy is not dectected, the analyses are slow \\
        Celestial & External calls are not considered, keywords storage and 
        memory are not recognized  \\
        Echidna &  Reentrancy is not dectected, no assetion mode for solidity 8\\
        Certora & Reentrancy is not dectected, not open-sources \\ 
        SolcVerify & It just gives warning, it does not provide a list of 
        transaction for breaking the given property\\
        Mythril & Just flat contracts are allowed \\ 
        Slither & Scan is based on the grammar, great amount of false negative \\ 
    \bottomrule
    \end{tabular}
\end{table*}

\begin{center}
\begin{table*}
    \caption{Strenghts}
        \label{tab:Strenghts}
        \begin{tabular}{cl}
        \toprule
            Tools  &  Strenghts \\
            \midrule
            Manticore & One of the mode cover the properties breaking and the scanner one covers the reentrancy\\
            SmartTest & It allows to set the specific vulenrability to look for  \\
            Celestial & Possibility to use different version of F*  \\
            Echidna &  Possibility to run in multiple modes with different grammar (tests or assertions breaking)\\
            Certora & Implements the library of Openzeppelin, SAS no installation needed \\ 
            SolcVerify & Intuitive specification language based on Annotations, it detects reentrancy\\
            Mythril & It deos not need specification, but still provide list of functions for breaking detected vulnerability  \\ 
            Slither & Easiest installation, fastest tool that we used \\ 
        \bottomrule
        \end{tabular}
    \end{table*}
\end{center}
Certora is the only one which provides a complete list of functions for breaking the rules, rather than just a warning. 
On the other hand, SolcVerify could detect the vulnerabilities involving external call functions, indeed reentrancy. A powerful aspects of this tool is its possibility to express 
loop invariants, the other ones do not allow it.
Considering the grammar for expressing the specifications, SolcVerify is the one which needs the least amount of lines of code, indeed it involves a notification language.

Celestial architecture encompasses two steps: the translation from celestial file to f* and then its verification. 
The python script converts the ".cel" in f*, used for the proof or unproof.
The provided file included the smart contract's source code plus the expressed specification. These are statements placed at the beginning of a function, otherwise, it is possible to create a sort of function, 
containing boolean formula, which is called by different specifications function with different parameters, it is useful for expressing the same specification for different purpose.  I consider it the one with more limitations, regarding solidity grammar and reentrancy, 
because it could not detect the reentrancy vunlenaribilities and the Cover protocol attack, because the keywords "storage" and "memory" are rejected. 

Certora is the only tool which is not open-source, for our purpose we adopted its free version.
Its specification language is described by its developers' group as "rule-based". It differs from the other two tools under this aspect, because this way gives more elasticity to the user and defines more specific cases.
The rule is composed of some function calls and it concludes with an assertion or more. The user is allowed to test a specific case, using "require" and the possibility to set up a proper environment. 
The preconditions, in this case, are expressed using the Solidity keyword "require" in the rule.

One of its strengths is the possibility to define the specifications we want to prove, without necessarily defining all the specifics for the rest of the functions.
On the other hand,  with Celestial and SolcVerify, we provided the specifications for all the functions for letting the tools work properly. 
Those could not prove the given properties, without the specifications for all the code.


\paragraph{Different stategies, similar grammar}
We considered two tools with similar grammar but implemented different analysis approaches: Manticore and Echidna. 
In both of the cases, we provided functions containing a boolean formula, which the tools try to break. From the results, 
we noticed that Echidna run faster and it worked for all the cases, but Manticore could cover the reentrancy vulnerabilities thanks to its changing architecture.

We encountered a common aspect between the grammar definition of the specification between the two tools Echidna and SmarTest regarding their "assertion" mode. 
Both of those require the user to write assertions and then these try to verify it or return the list of 
functions for breaking the rule. From our results, it is clear Echidna could obtain higher number of positive outomces and in less time rather than SmartTest.
