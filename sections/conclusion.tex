%% LaTeX2e class for student theses
%% sections/coclusion.tex
%% 
%% Karlsruhe Institute of Technology
%% Institute of Information Security and Dependability (KASTEL)
%%
%% Template by
%% Dr.-Ing. Erik Burger
%% burger@kit.edu
%%
%% Adaption by
%% Annika Vielsack
%% vielsack@kit.edu
%%
%% Version 1.0, 2021-07-03

\chapter{Conclusion}
\label{ch:Conclusion}
The security of smart contracts is a field which has been growing, 
as the popularity of blockchains and decentralised applications (web3, NFTs...) has been getting more spread.

This work presents the behaviour of analysis tools in a real-world context, 
providing insights into the process of definition of properties, usage of the tools and finding vulnerabilities.
It is addressed to present a possible approach in real situations.

Finally, a comparison of the selected tools is produced.
It deals with multiple aspects of those, starting with the installation and the user experience, involving the performance and effectiveness of the analysis.

The aim is to study the aspects of those in a situation as 
similar as possible to realty. 
The last chapters deal with their behaviour, understanding which approach is more effective. 
Moreover, this work covers even the process of definition of the properties.

This thesis provides the description of each tool, dealing with its performance, stating which typology of them had the best results. 
The involved attacks are presented, stating the properties per each of them. 

Security of smart contracts is fundamental for granting the correct behaviour of a decentralised application, but a limitation is a possibility 
of multiple threats involving the infrastructure of the system itself.
Possible attacks on the blockchain are not covered by this field. 
The network itself could be vulnerable and a user could risk getting the private key of the wallet stolen. 

In this field, the definition of the specifications is fundamental.
The focus of this thesis was on the vulnerable parts, for understanding which vulnerabilities were exploited by the attackers.
Dealing with the reviewed attacks, the way those were computed is known. 
The difficulties dealt with the definition of the properties and the correct usage of the tools. 
Reviewing a smart contract involves the additional task of trying to define all the possible vulnerabilities and writing those in form of specifications, 
for effective results from the tools providing custom analysis.

\section{Outlooks}
Further work can use the same tools, 
but analyse new attacks for evaluating the effectiveness of those tools in new real worlds cases.

New researches can adopt the same approach, involving new attacks and new tools.

It can provide the starting point for a code review, supporting developers 
and security analysts for the choice of tools for their experiments and tests, based on a
given selection criteria.

