%% LaTeX2e class for student theses
%% sections/introduction.tex
%% 
%% Karlsruhe Institute of Technology
%% Institute of Information Security and Dependability (KASTEL)
%%
%% Template by
%% Dr.-Ing. Erik Burger
%% burger@kit.edu
%%
%% Adaption by
%% Annika Vielsack
%% vielsack@kit.edu
%%
%% Version 1.0, 2021-07-03

\chapter{Introduction Template}
\label{ch:IntroductionTemplate}

%% -------------------
%% | Example content |
%% -------------------

This is the thesis template of the \emph{Application-oriented Formal Verification}
research group at the Institute of Information Security and Dependability (KASTEL) at KIT. It was adapted
from the thesis template of the SDQ research group.
Hence, for more information on the formatting of theses, you can still refer to
\url{https://sdqweb.ipd.kit.edu/wiki/Ausarbeitungshinweise} as well as your advisor.


In the following, we present some general recommendations.
These are non-binding, however, and should be harmonized with your advisor.

% \section{Content}
% \label{sec:content}
% 
% \begin{itemize}
% \item The typical thesis has (at the very least) three parts:
% the introduction, the main part, and the conclusion.

  
% \item Following points is mandatory for a thesis:
%   \begin{itemize}
%   \item The \textbf{introduction} containing the motivation, problem definition,
%     and your contribution.
% 
%     At the end of the introduction, every computer scientist reader should know,
%     which problem have you solved, how do you solve it, and why this is an
%     important contribution.
%        
%   \item The \textbf{preliminary work} consists the foundations on which your
%     thesis built on. 
% 
%   \item In contrast, \textbf{related work} contains foreign publications which
%     are in competition to your approach. This section can either be a chapter,
%     or be a section in the introduction.
% 
%     
%   \item The \textbf{conclusion} should summarize your work, especially what is
%     learned from the evaluation, and should point to possible future work.
% 
%     
%   \item The \texttt{abstract} is a half-page summary of your work. This abstract
%     is used in various places, e.g. announcing your thesis talk. Therefore, it
%     should be standalone understandable.
% 
%     
%   \end{itemize}
% FIXME: Where is the above from? This should be discussed!
% \end{itemize}

\section{Style and Typography}

It is advisable to choose and stick to a specific style guide for your thesis' language.
Especially for the English language, there are some commonly agreed style guides.
At times, the recommendations of one guide contradict the ones from another.
Therefore, be consistent in your choice!
The following two style guides are the most common ones:
\begin{itemize}
\item \emph{New Oxford Style Manual} for British English
\item \emph{The Chicago Manual of Style} for US-American English
\end{itemize}

If you are also interested in (some of the) choices considering the typography, you may
consult \emph{The Elements of Typographic Style} which comes with a plethora of explanations,
many of them already baked into \TeX{} and \LaTeX{}.

\subsection{Sections}
\begin{itemize}
\item Avoid single sections. If you have, e.g., a section ``1.1'' then you should also
  have at least a section ``1.2''.
\item Avoid having too small chapters or sections. Rather use \verb+\paragraph+
  to divide text into smaller chunks.
\end{itemize}

\subsection{Spacing and Indentation}
For separating parts of text in \LaTeX, please use two line breaks. They will then
be set with correct indentation. Do \emph{not} use:
\begin{itemize}
  \itemsep0em
\item \texttt{\textbackslash\textbackslash}
\item \texttt{\textbackslash parskip}
\item \texttt{\textbackslash vskip}
\end{itemize}
or other commands to manually insert spaces, since they break the layout of this template.

\subsection{Bibliography and References}

The bibliography in this template is already configured. This template is based
on \texttt{biblatex} and \texttt{biber}, which is preferred over the outdated
\hologo{BibTeX} software. Biber is a bibliography processor, and thus reads both
the \texttt{aux-} and \texttt{bib-}files to produce the bibliography. Biber should
come with your \LaTeX distribution. Please adjust your build environment if
necessary (see
\url{https://sdqweb.ipd.kit.edu/wiki/BibTeX-Literaturlisten#biblatex.2Fbiber})

For referencing literature in your bibliography, you should use the following
commands:
\begin{itemize}
\item \verb+\citet{KeYBook2016}+: \citet{KeYBook2016}\\[.5ex]%
Use this when you want to explicitly talk about a publication within a sentence.
This is especially sensible for publications that are of high relevance for your thesis.
\item \verb+\citep{KeYBook2016}+: \citep{KeYBook2016}\\[.5ex]%
Use this for (implicit) references to indicate that what you wrote is based on the cited reference.
The command can also be used for multiple references within one citation, separated by a comma (directly inside the command).
\end{itemize}

\subsection{Floats (Figures, Tables, ...)}
\begin{itemize}
\item Do not inline float environments such as tables, figures, listings,
algorithms etc.
Floats are elements that are automatically placed and optimized when compiling
the document.
\item Avoid using the options \texttt{H} or \texttt{h} for positioning floats.
\item A reference: The KIT logo is displayed in \autoref{fig:kitlogo}. (Use
  \verb+\autoref{label}+ for easy referencing.)
\item \textbf{For tables:} The \texttt{booktabs} package offers nicely typeset
  tables, as in \autoref{tab:atable}.
\item \texttt{For algorithms:} Algorithms can be nicely set by a variety
  of packages, e.g., \texttt{algorithm2e}, \texttt{algorithmicx}, etc.
\item \texttt{For source code:} The \texttt{lstlistings} or \texttt{minted}
package offer nicely typeset and colored listings for your source code.
\end{itemize}

\begin{figure}
  \centering
  \includegraphics[width=4cm]{logos/kitlogo_en_cmyk}
  \caption{KIT logo}
  \label{fig:kitlogo}
\end{figure}

\begin{table}
  \centering
  \begin{tabular}{r l}
    \toprule
    abc & def \\
    ghi & jkl \\
    \midrule
    123 & 456 \\
    789 & 0AB \\
    \bottomrule
  \end{tabular}
  \caption{A table}
  \label{tab:atable}
\end{table}

\subsection{Example: Mathematics}
One of the nice things about the Linux Libertine font is that it comes with
a math mode package.
\begin{displaymath}
  f(x)=\Omega(g(x))\ (x\rightarrow\infty)\;\Leftrightarrow\;
  \limsup_{x \to \infty} \left|\frac{f(x)}{g(x)}\right|> 0
\end{displaymath}

For \emph{definitions}, \emph{theorems}, \emph{proofs}, etc., please use
the respective environments.

\begin{theorem}[Pythagoras's theorem]
  In a right-angled triangle, the following holds:
  \begin{align*}
    a^{2}+b^{2}=c^{2}\enspace.
  \end{align*}

\end{theorem}

%% --------------------
%% | /Example content |
%% --------------------

\chapter{Introduction}
\label{ch:Introduction}


\section{Motivation}
\label{sec:Introduction:Motivation}

Blockchain represents one of the most popular trends in finance and computer science, 
during the last few years the number of investments has been growing exponentially. 
According \href{https://www.coingecko.com/}{CoinGeko}, the crypto market's value is standing around \$2 trillion.

Bitcoin can be considered the “father” of this technology. \citet{Bitcoin} depicted that in his paper, and in the early 2009,
it was effectively launched and the cryptocurrency Bitcoin was introduced. 
\href{https://www.coingecko.com/}{CoinGeko} states the value of Bitcoin around \$38,553.70 and its market capitalization more than  \$700 billions.

Many blockchain systems have been born with new capabilities, 
which have allowed them to fit many different use cases. The first, which allowed developers to 
code on top of itself, was Ethereum.
\citet{Ethereum} published its whitepaper in 2014, and in 2015 it was deployed.
The revolutionary aspect of Ethereum is the introduction of Smart Contract.
These are programs running on blockchain systems and give the developers the opportunity to interact directly 
with this new technology. 
The development of innovative and prominent applications is a consequence of their development, such as NFT marketplaces, music royalty tracking, supply chain and logistics monitoring, voting mechanism, 
cross-border payments, and many others.  

Interest in such a market has grown even among malicious attackers. 
Attacks such as the “Parity Wallet Hack” and the “Decentralized Autonomous Organization Attack” cost millions of dollars simply because of 
naive bugs in the smart contract code. Blockchain and smart contract technologies have multiple aims, but unfortunately, new applications 
based on them still contain bugs and multiple vulnerabilities, which cause 
several issues for the end-users. Most of the use of this technology relates to finance or certifications, therefore integrity, 
authentication and authorisation in transactions are mandatory. The research field behind blockchain technology is growing, as well as the one concerning 
its security and accordingly, many analysis tools were developed. 
These incorporate various strategies for performing the analyses, concerning the technical aspects of smart contracts, 
so these would work differently according to the object of the analysis. 

The topic that will be addressed in this thesis work is the analysis of smart contract security properties with the usage of tools. 
It involves the understanding of smart contracts properties and the comparison between different tools, 
providing insight regarding their behaviours in different contexts.


\section{Research Goals}
\label{sec:Introduction:ResearchGoals}
\paragraph{Research Question:} 
How do state-of-the-art analysis tools for Ethereum/Solidity perform (on different classes of properties/bugs)?

This thesis focuses on a dozen analytic tools, which we choose based on the type of analysis, trying to have a range of different typologies. 
We will test them on vulnerable smart contracts and figure out which properties are violated during real-world exploits. 
Furthermore, we are going to compare the tools, based on their performance, in particular, the criteria for the evaluation can cover the completeness of the analysis, the amount of found vulnerabilities and the number of false positive and negative. 
The execution time is crucial too, we want to understand how long it takes for finding a vulnerability. The time for the configuration and the report interpretability are parameters for defining how much a tool is user friendly. 

For answering the research question, we will give an answer to sub questions such as: 
\begin{itemize}
  \item How does a tool perform the analysis? 
  \item Which properties are relevant for smart contract security?
  \item Which ones have been violated in real-world exploits? 
  \item Which tools detect which class of vulnerabilities? 
\end{itemize}

\section{Research Approach}
\label{sec:Introduction:ResearchApproach}
Simil exposè

\section{Releted Works}
\label{sec:Introduction:ReletedWorks}
Simil exposè