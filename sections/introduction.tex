%% LaTeX2e class for student theses
%% sections/introduction.tex
%% 
%% Karlsruhe Institute of Technology
%% Institute of Information Security and Dependability (KASTEL)
%%
%% Template by
%% Dr.-Ing. Erik Burger
%% burger@kit.edu
%%
%% Adaption by
%% Annika Vielsack
%% vielsack@kit.edu
%%
%% Version 1.0, 2021-07-03
\chapter{Introduction}
\label{ch:Introduction}
Blockchain represents one of the most popular trends in finance and computer science.
It is a fully-distributed public ledger and a peer-to-peer platform.
Cryptography is the base for securely hosting applications, transferring digital currency/messages and
storing data. Bitcoin can be considered the “father” of this technology. \citet{Bitcoin} depicted that in his paper, in the early 2009,
it was effectively launched and the cryptocurrency Bitcoin was introduced. 
It works as a decentralised database.

Many blockchain systems have been born with new capabilities, 
which have allowed them to fit many different use cases. The first, which allowed developers to 
code on top of itself, was Ethereum.
\citet{Ethereum} published its whitepaper in 2014, and in 2015 it was deployed.
The revolutionary aspect of Ethereum is the introduction of Smart Contract.
These are programs running on blockchain systems and allow the developers to interact directly 
with this new technology. 
The development of innovative and prominent applications is a consequence of their development, such as NFT marketplaces, music royalty tracking, supply chain and logistics monitoring, voting mechanism, 
cross-border payments, decentralised finance and many others (\cite{BcUseCases}).  

The implementation of financial products enticed many investors during the last few years. 
The number of investments has been growing exponentially. 
According \citet{statista}, the crypto market's highest value reached around \$3 trillion in 2021.

Considering Bitcoin, the cryptocurrency with the highest market capitalization, 
\citet{CoinGeko} states its value around \$38,553.70 and its market capitalization more than \$700 billions.

Interest in such a valuable market has grown even among malicious attackers. 
Attacks such as the “Parity Wallet Hack” and the “Decentralized Autonomous Organization Attack” cost millions of dollars simply because of 
naive bugs in the smart contract code. Blockchain and smart contract technologies have multiple aims, but unfortunately, new applications 
based on them still contain bugs and multiple vulnerabilities, which cause 
several issues for the end-users. Most of the use of this technology relates to finance or certifications, therefore integrity, 
authentication and authorisation in transactions are mandatory. The research field behind blockchain technology is growing, as well as the one concerning 
its security and accordingly, many analysis tools were developed. 
These incorporate various strategies for performing the analyses, concerning the technical aspects of smart contracts, 
so these would work differently according to the object of the analysis. 

The topic that will be addressed in this thesis work is the comparison of security analysis tools for smart contracts, based on real-world exploits, so attacks that have happend during the recent years. 
It involves the understanding of smart contracts properties and the usage of different tools, 
providing insight regarding their behaviours in different contexts.

This thesis does not involve well-known benchmarks, with already studied vulnerabilities, 
but the tests are smart contracts involved in attacks. 
Our literature research faced off wide documentation of comparison of tools based on benchmarks and studied vulenrablities, 
but this work's ambition is to verify the effectiveness of the tools in real cases.

\section{Research Goals}
\label{sec:Introduction:ResearchGoals}
The main goal of this thesis can be summerized with the following research question:

\emph{How do state-of-the-art analysis tools for Ethereum/Solidity perform on real-world exploits?}

This thesis involeves eight security analysis tools, which are chosen based on a literature research and on the type of analysis, trying to have a range of different typologies.
Their analysis targets are smart contracts, involved in attacks, which have occured in the last two years (since 2020). 
One of the goals is the definition of the violated properties of those, understanding how they are computed by the attackers.

Furthermore, the comparison of the tools is based on a range of factors, these are some of the parameters used for providing a comparison and an evaluation of the tools.
\begin{itemize}
  \item the performance;
  \item the completeness of the analysis;
  \item the facility of usage, involving the amount of code to be provided;
  \item the amount of found vulnerabilities;
  \item the report interpretability;  
  \item the time for the configuration;
\end{itemize}  

The research question deals with different topics, which can be expressed with the following sub questions: 
\begin{enumerate}
  \item How does a tool perform the analysis? 
  \item Which properties have been violated in the real-world exploits? 
  \item Which vulnerabilities are the tools able to detect? 
  \item In which context a tool perform better?
\end{enumerate}


\section{Releted Works}
\label{sec:Introduction:ReletedWorks}
Nowadays multiple surveys and research work addressing smart contracts analysis have been published. 
The ones, we are interested in, deal with the review of vulnerabilities, description and comparison of tools and definition of new techniques for scanning those. 

The selection of tools was anticipated by research work. 
We picked those starting from surveys and papers, 
regarding comparison of multiple of tools, such as \citet{Survey1}, \citet{Survey2}, \citet{Survey3}, \citet{Survey4}, \citet{thesis}. 
These give a general overview and provide a comparison based on different aspects: type of installation, running mode or type of analysis. 
A taxonomy is provided as well. 
For having a deeper knowledge of every single tool, we considered their papers and documentation.

In this thesis, we involved automated tools (\citet{Slither}, \citet {Mythril}) and the ones which provide the possibility to run custom analysis.
The first ones have as targets vulnerabilities such as reentrancy, overflow/underflow, and gas exceptions; but they do not provide functional correctness guarantees. 
On the other hand, the second group try to solve these issues by providing more possibilities for modelling the analyses. 
We involved tools adopting formal verification (\citet{CertoraDocumentation}, \citet{SolcVerify}, \citet{CelestialPaper}) and fuzzing (\citet{Echidna}). 

The cited works provide a comparison based on well-known benchmarks, defined vulnerabilities or just on the specifications.
This work provides a comparison between the considered tools as well, 
but we tried to move a step forward.
Rather than considering defined vulnerabilities, we consider real-world exploits, which have happened in the last couple of years.

The considered tools are installed and run on real-world attacks; these are chosen based on their effectiveness and the damage, in terms of drawn liquidity.