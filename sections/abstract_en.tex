%% LaTeX2e class for student theses
%% sections/abstract_en.tex
%% 
%% Karlsruhe Institute of Technology
%% Institute of Information Security and Dependability (KASTEL)
%%
%% Template by
%% Dr.-Ing. Erik Burger
%% burger@kit.edu
%%
%% Adaption by
%% Annika Vielsack
%% vielsack@kit.edu
%%
%% Version 1.0, 2021-07-03

\Abstract
A blockchain is essentially a digital ledger of transactions that is duplicated and distributed across the entire network of computer systems on the blockchain.
Ethereum is considered a "smart contracts platform" which allows the development of smart contract applications.
Those are self-verifying programs running on top of the blockchain.

Smart contracts are public, distributed and immutable, consequently, developers cannot design security by layer.
The code must be examined for any potential weaknesses because they could lead to significant financial losses and harm public trust. 
The regulation of this field is not strict and smart contracts could hold millions of dollars.
For these reasons, malicious users computed a great number of attacks, adopting different strategies

To guarantee security, numerous tools have been created, and a large amount of data about vulnerabilities and detection techniques is continually being produced.
The security analysis tools implement the computer science security techniques, adapted for the blockchain context, such as fuzzing and symbolic execution.

This thesis is addressed to deepen the field of security in smart contract programming, written in Solidity: 
the most used and maintained programming language in this field.
This work presents a collection of tools and real-world exploits 

This work presents a collection of attacks and a collection of tools, selected after a literature research phase. 
Those are used for scanning real-world exploits that have occurred during the last two years (since 2020).
The aim is to provide a comparison between those based on real cases. 
The comparison involves parameters such as the time of the installation requirements, the time of execution, the configuration of settings and the amount of discovered vulnerabilities.
Furthermore, the tools are grouped based on their different characteristics, their typology, and even on their running mode. 


\chapter* {Acknowledgements}
I would like to express my gratitude to my supervisors Jonas Schiff, Prof. Bernhard Beckert and Prof. Valentina Gatteschi
for the oppurtunity to carry out this thesis and the support they provided to my.
I had written this thesis during my Erasmus in Karlsruhe, I would like to dedicate it to my friends who I met during this amazing experience:
Nico, oMaria, Michela, Anita, Luigi, Danny, Orhan, the portugues, las Sepias, los Chavales, las Perras.
